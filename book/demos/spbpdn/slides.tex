\title[Short title]{Statics preserving basis pursuit denoising}
\author{\small Aaron Stanton, Nasser Kazemi, and Mauricio D. Sacchi}
\institute{\small Signal Analysis and Imaging Group \\ Department of Physics \\ University of Alberta}
\date{\small \today \\ \vspace{0.5cm} \includegraphics[height=1cm,width=2cm]{logo}}

\addtobeamertemplate{frametitle}{}{%
\begin{textblock*}{100mm}(.85\textwidth,-0.9cm)
\includegraphics[height=1cm,width=2cm]{logo}
\end{textblock*}}

\maketitle

\begin{frame} \frametitle{Outline}
    \begin{itemize}
        \item Motivation
        \item Basis pursuit denoising (BPDN)
        \item Static-preserving BPDN
        \item Synthetic data example
        \item Real data example
    \end{itemize}
\end{frame}

\begin{frame} \frametitle{Motivation}
    \begin{itemize}
        \item Calculation of small static shifts using idea of sparsity
        \item Generalization of sparsity driven denoising methods ($l_1-l_2$ regression, BPD, LASSO, etc)
    \end{itemize}
\end{frame}

\begin{frame} \frametitle{Motivation}
\begin{columns}[c]
\column{2in} 
	\begin{center}
	\begin{figure} 
	\includegraphics[width=2in,height=3in]{synth/Fig/data_out1} \\ 
	\tiny Standard BPDN
	\end{figure} 
	\end{center}
\column{2in}
	\begin{center}
	\begin{figure} 
	\includegraphics[width=2in,height=3in]{synth/Fig/data_out2} \\
	\tiny Static preserving BPDN
	\end{figure}
	\end{center}
\end{columns}
\end{frame}

\begin{frame} \frametitle{BPDN}
\begin{center}
$
\quad{J}=\| {\bf{A}} {\bf{x}} - {\bf{d}} \|_2 + \lambda\|{\bf{x}}\|_1
$ 
\end{center}
\end{frame}

\begin{frame} \frametitle{SP-BPDN}
\begin{center}
$
\quad{J}=\| {\bf{S}}{\bf{A}} {\bf{x}} - {\bf{d}} \|_2 + \lambda\|{\bf{x}}\|_1
$ 
\end{center}
\end{frame}

\begin{frame} \frametitle{SP-BPDN}
\begin{center}
$
\quad{J}=\| {\bf{S}}{\bf{A}} {\bf{x}} - {\bf{d}} \|_2 + \lambda\|{\bf{x}}\|_1
$

\vspace{1cm}

$\nabla_xJ=0$ FISTA

\vspace{1cm}

$\nabla_SJ=0$ Maximum Likelyhood Estimator

\end{center}
\end{frame}

\begin{frame} \frametitle{Pseudocode: BPDN}
\begin{algorithm}[H]
\renewcommand{\thealgorithm}{}
\tiny
\caption{Basis pursuit denoising}\label{BPDN}
\begin{algorithmic}
\Procedure{BPDN}{d,$N_{iter},t,\alpha$}\Comment{solves $||Ax-d||^2_2 + \lambda||x||_1$}
   \State $x=A^Td$ 
   \State $d^{pred}=d$ 
   \While{$iter\leq N_{iter}$}
      \State $r=d^{pred} - d$ 
      \State $g=A^Tr$
      \State $x = x - 2tg$
      \State $soft(x,\alpha)$
      \State $d^{pred} = Ax$
      \State $iter++$
   \EndWhile
   \State \textbf{return} $x$\Comment{d = Ax}
\EndProcedure
\end{algorithmic}
\end{algorithm}
\end{frame}

 {\color{red}}

\begin{frame} \frametitle{Pseudocode: SP-BPDN}
\begin{algorithm}[H]
\renewcommand{\thealgorithm}{}
\tiny
\caption{Statics preserving basis pursuit denoising}\label{SPBPDN}
\begin{algorithmic}
\Procedure{{\color{red}SP-}BPDN}{d,$N_{iter},t,\alpha$}\Comment{solves $||SAx-d||^2_2 + \lambda||x||_1$}
   \State ${\color{red}\delta=0}$ 
   \State $x=A^T{\color{red}S(\delta)^T}d$ 
   \State $d^{pred}=d$ 
   \While{$iter\leq N_{iter}$}
      \State $r={\color{red}S(\delta)}d^{pred} - d$ 
      \State $g=A^T{\color{red}S(\delta)^T}r$
      \State $x = x - 2tg$
      \State $soft(x,\alpha)$
      \State $d^{pred} = {\color{red}S(\delta)}Ax$
      \State ${\color{red}\delta=xcorr(d,d^{pred})}$
      \State $iter++$
   \EndWhile
   \State \textbf{return} $x$\Comment{$d = {\color{red}S(\delta)}Ax$}
\EndProcedure
\end{algorithmic}
\end{algorithm}
\end{frame}

\begin{frame} \frametitle{ }
	\begin{center}
        \LARGE {\bf Radon Operator}
	\end{center}
\end{frame}

\begin{frame} \frametitle{Gather with noise \& statics}
	\begin{center}
	\begin{figure} 
	\includegraphics[height=3in]{radon_figs/original_data-eps-converted-to.pdf} \\
	\end{figure} 
	\end{center}
\end{frame}

\begin{frame} \frametitle{Radon Panels}
\begin{columns}[c]
\column{2in} 
	\begin{center}
	\begin{figure} 
	\includegraphics[height=3in]{radon_figs/high_resolution_radon_without_statics-eps-converted-to.pdf} \\ 
	\tiny Standard Radon panel
	\end{figure} 
	\end{center}
\column{2in}
	\begin{center}
	\begin{figure} 
	\includegraphics[height=3in]{radon_figs/high_resolution_radon_with_statics-eps-converted-to.pdf} \\
	\tiny Static preserving Radon panel
	\end{figure}
	\end{center}
\end{columns}
\end{frame}

\begin{frame} \frametitle{Denoised gathers}
\begin{columns}[c]
\column{2in} 
	\begin{center}
	\begin{figure} 
	\includegraphics[height=3in]{radon_figs/estimated_data_without_statics-eps-converted-to.pdf} \\ 
	\tiny Standard BPD
	\end{figure} 
	\end{center}
\column{2in}
	\begin{center}
	\begin{figure} 
	\includegraphics[height=3in]{radon_figs/estimated_data_with_statics-eps-converted-to.pdf} \\
	\tiny Static preserving BPD
	\end{figure}
	\end{center}
\end{columns}
\end{frame}

\begin{frame} \frametitle{ }
	\begin{center}
        \LARGE {\bf F-K operator:\\Synthetic gather}
	\end{center}
\end{frame}

\inputdir{synth}
\begin{frame} \frametitle{True data}
    \plot{data}{width=2in,height=3in}{}
\end{frame}

\begin{frame} \frametitle{Statics (time variant +/- 5ms)}
    \plot{data_statics}{width=2in,height=3in}{}
\end{frame}

\begin{frame} \frametitle{Statics and Noise}
    \plot{data_statics_noise}{width=2in,height=3in}{}
\end{frame}
\begin{frame} \frametitle{Output: BPDN}
    \plot{data_out1}{width=2in,height=3in}{}
\end{frame}
\begin{frame} \frametitle{Output: SP-BPDN}
    \plot{data_out2}{width=2in,height=3in}{}
\end{frame}

\begin{frame} \frametitle{ }
	\begin{center}
        \LARGE {\bf F-K operator:\\Alaska North Slope}
	\end{center}
\end{frame}

\inputdir{northslope}
\begin{frame} \frametitle{True data}
    \plot{data}{width=5in,height=3.2in}{}
\end{frame}
\begin{frame} \frametitle{Output: BPDN}
    \plot{data_out1}{width=5in,height=3.2in}{}
\end{frame}
\begin{frame} \frametitle{Output: SP-BPDN}
    \plot{data_out2}{width=5in,height=3.2in}{}
\end{frame}


\begin{frame} \frametitle{Conclusion}
    \begin{itemize}
        \item Static shifts reduce sparsity of the data
        \item We can account for this in an alternating minimization scheme
        \item Data are effectively denoised while preserving statics  
        \item Could be an advantage in areas with complicated structure
    \end{itemize}
\end{frame}


