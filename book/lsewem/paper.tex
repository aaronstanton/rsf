
\title{Least squares migration of multicomponent seismic data}

\author{Aaron Stanton and Mauricio D. Sacchi}

\address{
University of Alberta, \\
Department of Physics, \\
4-183 CCIS, \\
Edmonton AB T6G 2E1 \\
aaron.stanton@ualberta.ca}

\footer{}
\lefthead{Stanton and Sacchi}
\righthead{Multicomponent LSM}

\maketitle

\ms{GEO-2014}

\begin{abstract}
  Write this after writing the paper. Be sure to use active voice. Be informative.
\end{abstract}

\section{Introduction}
Least squares migration aims to find the inverse of a modelling operator such that the migrated data fits the observations in a least squares sense. It improves migration in the presence of noise, missing data, or poor subsurface illumination Citation. Elastic migration carries the same objectives as the acoustic case, but suffers from the additional complications of polarization and stronger aliasing of the S-wave mode. 

\section{Theory}

\subsection{Migration of elastic data}
Much work has been done on the migration of elastic data. \cite{chang1987elastic} apply Reverse Time Migration to elastic data, producing images for the horizontal and vertical components. OTHERCITATION apply Helmholtz decomposition at the imaging condition to produce separate PP and PS images.

MORE ON RTM methods advantage: two-way wave propagation, mode conversion at each interface. Disadvantage: computational cost, especially for S-waves which require very small grid size and time step

KIRCHOFF \cite{doi:10.1190/1.1444783}: Advantage: Computational cost low. Disadvantage: high frequency approximation.

WAVE EQN BASED METHODS: 





A common point of all elastic migration techniques is the need for seperation of data into potentials. The next section discusses seperation in isotropic media.

\subsection{wavefield seperation}

For isotropic media, the relationship between two a component wavefield and P and S-wave potentials is given by the divergence and curl respectively

\begin{align}
p = \frac{\partial u_x}{\partial x} + \frac{\partial u_z}{\partial z}\\
sv = \frac{\partial u_z}{\partial x} - \frac{\partial u_x}{\partial z} 
\end{align}

The derivatives in this equation can be evaluated using finite differences or via Fourier transform 

\begin{align}
P(\omega,k_x) = i k_x U_x(\omega,k_x) + i k_z(v_s) U_z(\omega,k_x)\\
SV(\omega,k_x) = i k_x U_z(\omega,k_x) + i k_z(v_p) U_x(\omega,k_x)
\end{align}

\citep{etgen1988}.

This can be written as a linear operator

\begin{equation}
\begin{bmatrix}
	p\\
	s\\
\end{bmatrix}
=\mathbf{H}
\begin{bmatrix}
	u_x\\
	u_z\\
\end{bmatrix}
\end{equation}

where

\begin{equation}
\mathbf{H} = \mathbf{F_x}^{-1}
\begin{bmatrix}
	ik_x & ik_z(v_s)\\
 -ik_z(v_p) & ik_x\\
\end{bmatrix}\mathbf{F_x}
\end{equation}

Alternatively, to recompose potentials into wavefield components we apply

\begin{equation}
\begin{bmatrix}
	d_x\\
	d_z\\
\end{bmatrix}
=\mathbf{H}^{-1}
\begin{bmatrix}
	p\\
	s\\
\end{bmatrix}
\end{equation}

where
 
\begin{equation}
\mathbf{H}^{-1} = \mathbf{F_x}^{-1} \frac{1}{k_x^2 + k_z(v_p)k_z(v_s)}
\begin{bmatrix}
	-ik_x & ik_z(v_s)\\
	-ik_z(v_p) & -ik_x\\
\end{bmatrix}\mathbf{F_x}
\end{equation}

Give small demonstration of the operator $\mathbf{H}$, it works well on noise-free fully sampled data.



Under the Born approximation a scatterer is considered as a perturbation within a smooth background medium.  

Derive split step migration from the Born approximation. Refer to Kaplan et al (2010). 

Finally, show in operator form:

In Split-Step migration the propagator is given by:

\begin{equation}
\mathbf{P} = \mathbf{B}\mathbf{F}_x\mathbf{C}\mathbf{F}_x^{-1}
\end{equation}

Where $\mathbf{C}$ is a phase-shift operator \citep{gazdag1978wave}, and $\mathbf{B}$ is the Split-Step correction \citep{Stoffa01041990}. 

\begin{equation}
\mathbf{L}^* = \mathbf{A}\mathbf{P} \mathbf{\Phi_s}\mathbf{P} \mathbf{\Phi_g}
\end{equation}

$\mathbf{A}$ is the imaging condition.

If the imaging condition is taken to be migration to zero offset then $\mathbf{A}$ is simply a row vector that multiplies the source wavefield with the complex conjugate of the receiver wavefield and sums over frequency. This is the zero lag cross correlation of the source and receiver wavefields. 

Figure ??? shows an example of applying the migration operator to a single shot over a three layer model. For migration it is best to use the decompostion operator $\mathbf{H}$ directly, rather than the adjoint of the recomposition operator, $(\mathbf{H}^-1)^T$, which is used in the inversion process.

Angle gathers: 

Imaging to zero offset:

\begin{equation}
I(x,z) = \sum _s \sum _\omega q_-(x,z,\omega,s) q_+(x,z,\omega,s)*  
\end{equation}

Imaging with subsurface offset:

\begin{equation}
I(x,h,z) = \sum _s \sum _\omega q_-(x-h,z,\omega,s) q_+(x+h,z,\omega,s)*  
\end{equation}

\citep{rickett2002offset}.


\subsection{modelling elastic data}

Once you develop the operator notation for migration, take the adjoint of this for the forward operator.



\subsection{least squares inversion}


The cost function for elastic least squares migration with quadratic regularization is
\begin{equation}
J = || \mathbf{H}^{-1}\mathbf{L}\mathbf{m} - \mathbf{d} ||^2_2 + \mu ||\mathbf{m}||^2_2
\end{equation}
where $J$ is the cost to be minimized, $\mathbf{H}^{-1}$ is the Helmholtz recomposition operator, $\mathbf{L} = [\mathbf{L_{pp}} \; \mathbf{L_{ps}}]$ is the modelling operator, $\mathbf{m}=[\mathbf{m_{pp}} \; \mathbf{m_{ps}}]^T$ are the migrated data, and $\mathbf{d}=[\mathbf{d_{x}} \; \mathbf{d_{z}}]^T$ are the x and z components of the input data.

The foward and adjoint operators used in this inversion assume that mode conversion only occurs at the reflector. Alternatively the decomposition/ recompostion operator can be built directly into the migration operator, $L$, where it is applied at each depth step. \cite{bale2006} uses propagator matrices derived from the Christoffel equation to decompose and recompose anisotropic wavefields at individual depth steps for wave equation migration.

\subsection{preconditioning}

Rational for preconditioning, FK fan filtering of angle gathers, demonstration on 1 angle gather.

\begin{equation}
J = || \mathbf{H}^{-1}\mathbf{L}\mathbf{m} - \mathbf{d} ||^2_2 + \mu ||\mathbf{D}\mathbf{m}||^2_2
\end{equation}

Let $\mathbf{z}=\mathbf{D}\mathbf{m}$, then by a change of variables
\begin{equation}
J = || \mathbf{H}^{-1}\mathbf{L}\mathbf{D}^{-1}\mathbf{z} - \mathbf{d} ||^2_2 + \mu ||\mathbf{z}||^2_2
\end{equation}
we can simply choose $\mathbf{D}^-1$ to be any operator that emphasizes good features in our model. In our case we chose an FK fan filter acting on angle gathers.

\section{Examples}

\begin{figure}
\centering
\includegraphics[width=\textwidth]{simple/model/Fig/v.pdf}
\caption{Three layer P and S-wave velocity models.}
\label{fig:v_simple}
\end{figure}

\begin{figure}
\centering
\includegraphics[width=\textwidth]{simple/fdmod/Fig/d.pdf}
\caption{X and Z components for a single shot located at x=3500m modelled using the velocity model shown in figure \ref{fig:v_simple}.}
\label{fig:d_simple}
\end{figure}

\begin{figure}
\centering
\includegraphics[width=\textwidth]{simple/adjoint/Fig/m_gather.pdf}
\caption{PP and PS gathers for 50 shots migrated using the model shown in figure \ref{fig:v_simple}.}
\label{fig:m_gather_simple}
\end{figure}

\begin{figure}
\centering
\includegraphics[width=\textwidth]{simple/adjoint/Fig/m_angle.pdf}
\caption{PP and PS constant angle ($\tan \theta = 0.5$) section for 50 shots migrated using the model shown in figure \ref{fig:v_simple}.}
\label{fig:m_angle_simple}
\end{figure}


\begin{figure}
\centering
\includegraphics[width=\textwidth]{simple/fkfilter/Fig/m_gather_filter.pdf}
\caption{Example of FK filter applied to the angle gather shown in figure \ref{fig:m_gather_simple}.}
\label{fig:m_gather_simple}
\end{figure}

%\plot{m_gather}{width=0.7\textwidth}{Angle gather at x=3500m.}
%\plot{m_angle}{width=0.7\textwidth}{Stack.}

\begin{figure}
\centering
\includegraphics[width=\textwidth]{marmousi2/model/Fig/v.pdf}
\caption{P and S-wave velocities for a portion of the Marmousi2 velocity model.}
\label{fig:v_marmousi2}
\end{figure}

\begin{figure}
\centering
\includegraphics[width=\textwidth]{marmousi2/data/Fig/u.pdf}
\caption{X and Z components for a shot gather located at x=3500m modelled using the Marmousi2 velocity model.}
\label{fig:u_marmousi2}
\end{figure}
\section{Discussion}

discussion text.

\section{Conclusions}

conclusions.

\section{Acknowledgments}

We thank the sponsors of the Signal Analysis and Imaging Group (SAIG) for supporting this research. Computational resources were provided by WestGrid and Compute Canada Calcul Canada. The reproducible numeric examples use the Madagascar open-source package.

%\onecolumn

\bibliographystyle{seg}
\bibliography{SEG,lsewem}
